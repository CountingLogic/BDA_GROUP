
\documentclass{article}
\usepackage[T1]{fontenc} 
\usepackage[utf8]{inputenc}
\usepackage[english]{babel}
\usepackage{comment}
\usepackage{amsmath}
\usepackage{amsfonts}
\usepackage[stretch=10]{microtype}
\usepackage{hyperref}
\usepackage{pythonhighlight}
\usepackage{ dsfont }
\usepackage{xcolor}
\title{Chapter 1}
\author{Sagar Malhotra}


\begin{document}
\maketitle
\newpage
\subsection*{Prelude about Fundamentals of Bayesian Inference}
\subsubsection*{What is Bayesian Inference?}
Bayesian Inference is the process of fitting a probability model to a set of data and summarising the result by a probability distribution on the parameters of the model and on unobserved quantities such as predications for new observations.

\subsubsection*{Components of BDA}
\begin{itemize}
    \item Set up a full probability model: A joint Probability Distribution for all obdervables and unobservable quantities in a problem. 
    \item Conditioning on observed data : calculating and interpreting the appropriate posterior distribution—the conditional probability distribution of the unobserved quantities of ul- timate interest, given the observed data
\noindent\fbox{%
    \parbox{\textwidth}{%
    I AM CLUELESS ABOUT THE STEP 3 
   }%
}
    \item Evaluating the fit of the model and the implications of the resulting posterior distribution: how well does the model fit the data, are the substantive conclusions reasonable, and how sensitive are the results to the modeling assumptions in step 1? In response, one can alter or expand the model and repeat the three steps.



\end{itemize}
\end{document}

